%-------------------------------------------------------------------------------
%	SECTION TITLE
%-------------------------------------------------------------------------------
\cvsection{Work Experience}


%-------------------------------------------------------------------------------
%	CONTENT
%-------------------------------------------------------------------------------
\begin{cventries}

%---------------------------------------------------------
  \cventry
    {Senior Software Engineer \& Language Modeling} % Job title
    {SoundHound AI, Inc.} % Organization
    {Santa Clara, CA, USA} % Location
    {Apr. 2017 - Apr. 2020} % Date(s)
    {
      \begin{cvitems} % Description(s) of tasks/responsibilities
        \item Language Modeling (LM) for Automatic Speech Recognition (ASR)
        \begin{itemize}
          \item Leveraged MapReduce (Hadoop) and Java to train extensive language models, enhancing ASR accuracy through grammatical insights.
          \item Specialized in Japanese and Mandarin Chinese LMs, achieving industry-leading accuracy rates for Japanese speech recognition.
          \item Delivered innovative language models to cater to diverse customer requirements, with a focus on handling phonetic information challenges.
          \item Pioneered text segmentation methods, crucial for Asian languages, and optimized vocabulary and segmentation dimensions.
        \end{itemize}
        \item Diagnostic Tools Development
        \begin{itemize}
          \item Crafted a RESTful backend service for language penalty diagnostics using HBase, Jetty, and Java.
          \item Employed MapReduce, scripting languages, and command-line tools to diagnose ASR discrepancies by scrutinizing extensive training and intermediate datasets.
          \item Devised efficient algorithms to process and analyze large volumes of data.
        \end{itemize}
        \item Dashboard Creation and Management
        \begin{itemize}
          \item Developed a user-friendly dashboard frontend with Bootstrap and D3.js, showcasing charts and tables for streamlined capacity planning and quick data visualization.
          \item Constructed the dashboard backend using MySQL and Flask with Python.
          \item Designed Python and shell scripts to fetch server-side data, storing it systematically in databases.
          \item Introduced multiple Slack bots to ensure prompt event notifications.
        \end{itemize}
        \item Team Expansion in Tokyo
        \begin{itemize}
          \item Innovated a programming competition approach to identify top-tier software engineering talent, successfully expanding the Tokyo office team from 2 to approximately 20 members.
        \end{itemize}
      \end{cvitems}
    }

%---------------------------------------------------------
  \cventry
    {Software Engineer} % Job title
    {NeuroLeap, Inc.} % Organization
    {Santa Clara, CA, USA} % Location
    {Jun. 2014 - Apr. 2017} % Date(s)
    {
      \begin{cvitems} % Description(s) of tasks/responsibilities
        \item \textbf{Developed an Assessment Application for iPad}: Streamlined the evaluation process for therapists working with children in schools.
    \begin{itemize}
        \item Utilized both \textbf{Objective-C and Swift} to ensure robust and efficient app performance.
        \item Targeted the app at therapists evaluating children's learning disabilities, providing a digital solution to a previously manual and paper-heavy process.
        \item Dramatically \textbf{reduced the workload} for therapists, transitioning them from cumbersome paper tests to a more efficient, digital-first approach.
        \item Integrated standard frameworks alongside innovative features such as:
        \begin{itemize}
            \item \textbf{Automated Question Updates}: Seamlessly fetched and updated questions from online spreadsheets, ensuring the app's content was always current without the need for constant manual updates.
            \item \textbf{Embedded WebView}: Facilitated rich report generation directly within the app, providing comprehensive insights and overviews for each assessment.
        \end{itemize}
    \end{itemize}
      \end{cvitems}
    }

%---------------------------------------------------------
\end{cventries}
